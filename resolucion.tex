\documentclass[a4paper]{article}

\usepackage[spanish]{babel} % Le indicamos a LaTeX que vamos a escribir en espa�ol.
\usepackage[latin1]{inputenc} % Permite utilizar tildes y e�es normalmente
%\usepackage{framed}
\input{Algo1Macros}% Macros especificas para especificar problemas en AyEDI
\usepackage{listings}
\usepackage{color}
\usepackage{caption}
\usepackage{amssymb}

\definecolor{mygreen}{rgb}{0,0.6,0}
\definecolor{mygray}{rgb}{0.5,0.5,0.5}
\definecolor{mymauve}{rgb}{0.58,0,0.82}
\definecolor{myred}{rgb}{0.8,0.1,0.2}

% Configuracion de listings
\lstset{
  language=C++,                   % choose the language of the code
  numbers=left,                   % where to put the line-numbers
  numbersep=10pt,                 % how far the line-numbers are from the code
  backgroundcolor=\color{white},  % choose the background color. You must add \usepackage{color}
  tabsize=4,                      % sets default tabsize to 2 spaces
  breaklines=true,                % sets automatic line breaking
  breakatwhitespace=true,         % sets if automatic breaks should only happen at whitespace
  title=\lstname,                 % show the filename of files included with \lstinputlisting;
  showstringspaces=false,          % underline spaces within strings only
  basicstyle=\ttfamily,
  commentstyle=\color{mygreen},    % comment style
  keywordstyle=\color{blue},       % keyword style
  numberstyle=\tiny\color{mygray}, % the style that is used for the line-numbers
  stringstyle=\color{mymauve},     % string literal style
}

\DeclareCaptionFont{white}{ \color{white} }
\DeclareCaptionFormat{listing}{
  \colorbox[cmyk]{0.43, 0.35, 0.35,0.01 }{
    \parbox{\textwidth}{\hspace{15pt}#1#2#3}
  }
}
\captionsetup[lstlisting]{ format=listing, labelfont=white, textfont=white, singlelinecheck=false, margin=0pt, font={bf,normalsize} }

\newcommand{\comen}[2]{%
\begin{framed}
\noindent \textsf{#1:} #2
\end{framed}
}

\begin{document} % Todo lo que escribamos a partir de aca va a aparecer en el documento.

\section{C�digo}

\lstinputlisting{src/fatFood.cpp}
\lstinputlisting{src/tipos.h}
\lstinputlisting{src/combo.h}
\lstinputlisting{src/combo.cpp}
\lstinputlisting{src/pedido.h}
\lstinputlisting{src/pedido.cpp}
\lstinputlisting{src/local.h}
\lstinputlisting{src/local.cpp}
\lstinputlisting{src/auxiliar.h}
\lstinputlisting{src/auxiliar.cpp}

\section{Demostraciones}


\subsection{Demostraci�n del problema 'elVagonetaL'}


$ \\ P_c : vago==0 \land n==|empYdes| \land i==1 \land empYdes==[(e, descansoMasLargo(this,e)) \: | \: e \selec empleados(this)) ]$
$\newline$
$  Q_c : (\forall j \selec [0..|empYdes|) ) \: snd(empYdes_j) \leq snd(empYdes_{vago}) \land n == |empYdes|$
$\newline$
$  I : 0 \leq i \leq n \land n==|empYdes| \land (\forall j \selec [0..i) ) \: snd(empYdes_j) \leq snd(empYdes_{vago}) $
$\newline$
$ B: i < n $
$\newline$
$  cota : n-1$
$\newline$
$ fv: i $

$\\ P_c \land B \implica I $:

    Por $P_c$ sabemos que $i==1$ y por $B$ sabemos que $i < n$, lo que implica $0 \leq i \leq n$.

    Tambi�n, dado que $i==1$, lo que nos queda que
        $$ (\forall j \selec [0..i) ) \: snd(empYdes_j) \leq snd(empYdes_{vago}) $$
    \indent es lo mismo que
        $$ (\forall j \selec [0] ) \: snd(empYdes_j) \leq snd(empYdes_{vago}) $$
    \indent que es lo mismo que preguntar
        $$ snd(empYdes_0) \leq snd(empYdes_{vago}) $$
    \indent y como sabemos por $P_c$ que $vago == 0$, nos queda 
        $$ snd(empYdes_0) \leq snd(empYdes_0) $$
    \indent que es una tautolog�a. Entonces puedo decir que $ P_c \land B \Rightarrow I $.


$\\ I \land \neg B \implica Q_c $:

    Por $\neg B$ sabemos que $i\geq n$ y por $I$ sabemos que $i \leq n$, entonces $i==n$.

    Tambi�n sabemos por $I$ que $n==|empYdes|$, lo que nos queda que $i==|empYdes|$.

    Por $I$ sabemos que
        $$ (\forall j \selec [0..i) ) \: snd(empYdes_j) \leq snd(empYdes_{vago}) $$ 
    \indent y por lo dicho reci�n sobre $i$, podemos decir que 
        $$ (\forall j \selec [0..|empYdes|) ) \: snd(empYdes_j) \leq snd(empYdes_{vago}) $$ 
    \indent y entonces nos queda que $I \land \neg B \implica Q_c $


$\\ I \land cota < fv \implica \neg B $:

    Esto es lo mismo que decir que si se cumple el invariante y $n-1 < i$ entonces se niega la guarda.

    Sabemos que $n-1 < i$ y por el invariante $i \in [0..n]$, por lo tanto como $i$ es entero, $i == n$.

    Entonces, como $i==n$ y $n \nless n$, la guarda no se cumple.


\subsection{Demostraci�n del problema 'candidatosAEmpleadosDelMesL'}


$ \\ P_c : emp==empleadoConMasVentas(this) \land n==|emp| \land i==0 \land res==[] \\ $
    \indent $\land \: maxCombos = |combosVendidosPorElEmpleado(this,empleadosConMasCombos(this,emp)_0)| $
$\newline$
$  Q_c : i==n \land res == [e \: | \: e \selec [0..|emp|), |combosVendidosPorElEmpleado(this,e)| == maxCombos] $
$\newline$
$  I : 0 \leq i \leq n \land res == [e \: | \: e \selec [0..i), |combosVendidosPorElEmpleado(this,e)| == maxCombos] $
$\newline$
$ B: i < n $
$\newline$
$  cota : n-1$
$\newline$
$ fv: i $

$ \\ P_c \land B \implica I $:

    Por $P_c$ sabemos que $i==0$ por lo tanto vale que $0 \leq i \leq n$ pues $0 \leq n$ por ser $n$ la longitud de una lista. 

    Tambi�n, dado que $i==0$, $[0..i)$ es lo mismo que $[]$, entonces 
        $$ [\: e \: | \: e \selec [0..i), |combosVendidosPorElEmpleado(this,e)| == maxCombos \:] == []$$
    \indent que es igual a $res$ por $P_c$. Luego, puedo decir que $ P_c \land B \Rightarrow I \therefore $


$\\ I \land \neg B \implica Q_c $: 

    Por $\neg B$ sabemos que $i \geq n$ y por $I$ sabemos que $i \leq n$, entonces $i==n$.

    De $I$ sabemos que
        $$ res == [e \: | \: e \selec [0..i), |combosVendidosPorElEmpleado(this,e)| == maxCombos] $$
    \indent y por lo dicho reci�n sobre $i$, como $n == |emp|$ podemos decir que
        $$ res == [e \: | \: e \selec [0..|emp|), |combosVendidosPorElEmpleado(this,e)| == maxCombos] $$
    \indent y entonces nos queda que $I \land \neg B \implica Q_c $

$\\ I \land cota < fv \implica \neg B $:

    Por $I$ sabemos que $i \in [0..n]$ y si $cota < fv$, esto quiere decir que $n-1 < i$, que es lo mismo que $n \leq i$ pues $n$ e $i$ son numeros naturales.

    Entonces si $i \leq n$ y $n \leq i$, resulta $i==n$ con lo cual negamos la $B$ pues este dec�a que $i<n$.

\subsection{Demostraci�n del problema 'agregarComboAlPedidoL'}


$ \\ P_c : i==0 \land j==0 \land m==|ventasL(this)|\land n==pre(n) \land c==pre(c)$
$\newline$
$  Q_c : i==m \land p==ventasL(this)[j] \land numeroP(ventasL(this)[j]==n$
$\newline$
$  I : 0 \leq i \leq m \land ((p==ventasL(this)[j] \land numeroP(ventasL(this)[j]==n \land (\exists v \leftarrow ventas(this)) v.numeroP()==n) \lor (\forall k \leftarrow [0..i)) ventasL(this).numeroP \neq n) $
$\newline$
$ B: i < m $
$\newline$
$  cota : m-1$
$\newline$
$ fv: i $

$ \\ P_c \land B \implica I :$ Por $P_c$ sabemos que $i==0$ y por $B$ sabemos que $i < m$, lo que implica $0 \leq i \leq m$. Tambi�n, dado que $i==1$, lo que nos queda que $(\forall j \selec [0..i) ) 	snd(empleadosL(this)_j \leq snd(empleadosL(this)_{vago})$ es lo mismo que $(\forall j \selec [0] ) 	snd(empleadosL(this)_j) \leq snd(empleadosL(this)_{vago}) $. Ahora, preguntar eso es lo mismo que preguntar $snd(empleadosL(this)_0) \leq snd(empleadosL(this)_{vago})$, y como sabemos por $P_c$ que $vago == 0$, nos queda $\newline snd(empleadosL(this)_0) \leq snd(empleadosL(this)_0)$; que es una verdad (en particular, son iguales). Entonces puedo decir que $ P_c \land B \Rightarrow I :$.
$\newline$
$\\ I \land \neg B \implica Q_c :$ Por $\neg B$ sabemos que $i\geq n$ y por $I$ sabemos que $i \leq n$, entonces $i==n$. Tambi�n sabemos por $I$ que $n==|empYdes|$, lo que nos queda que $i==|empYdes|$. Por $I$ sabemos que $(\forall j \selec [0..i) ) 	snd(empleadosL(this)_j) \leq snd(empleadosL(this)_{vago})$ y por lo dicho reci�n sobre $i$, podemos decir que $(\forall j \selec [0..i) ) 	snd(empleadosL(this)_j) \leq snd(empleadosL(this)_{vago})$ es lo mismo que $(\forall j \selec [0..|empYdes|) ) 	snd(empleadosL(this)_j) \leq snd(empleadosL(this)_{vago})$; y entonces nos queda que $I \land \neg B \implica Q_c $
$\newline$
$\\ I \land cota < fv \implica \neg B :$ Esto es lo mismo que decir que si se cumple el invariante y $n-1 < i$ entonces se niega la guarda. Sabemos que $i \in [0..n]$, lo que nos deja un �nico caso para que $n-1<i$, dados que son enteros, y es que sean iguales. Entonces, como $i==n$ y $n \nless n$, la guarda no se cumple.

\end{document} %Termin�!

